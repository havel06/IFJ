\section{Rozdělení práce a práce v týmu}
Vzhledem k dobrým přátelským vztahům naší skupiny, tedy i bezproblémové
komunikaci, jsme se rozhodli zvolit iterativní vývoj. Každý člen týmu pracoval
na části, která byla zrovna potřeba vytvořit či opravit. Práci jsme se však
snažili rovnoměrně rozprostřít mezi všechny členy týmu, čemuž odpovídá i
rovnoměrné rozložení udělených bodů.

\subsection{Vývojový cyklus}
Jako základním kamenem společné práce se stal verzovací program \texttt{git}.
Jak již bylo zmíněno, na celém projektu jsme pracovali inkrementálně a od
jednodušších částí. Vytvořili jsme základní kostru všech částí a postupně
přidávali všechny potřebné funkce. Zároveň byla jasná potřebná posloupnost
implementace funkcionalit v různých částech, tedy nejdříve podpora v lexeru,
následně v parseru, poté v analyzéru a nakonec v samotném generátoru kódu.

Samozřejmě s tímto postupem je spojena i možnost, že při implementaci nové
funkcionality a změny funkcí pro její podporu se vytvoří chyba ve
funkcionalitách jiných. Z tohoto důvodu jsme využili automatické testy, abychom
případné chyby odhalili.

Komunikace celého týmu probíhala primárně ve formě osobních schůzek, kdy jsme
stav celého projektu prodiskutovali a vyřešili případné vetší problémy. Avšak
využití online platforem pro komunikaci bylo pro rychlý vývoj nutností. Zde
byly diskutovány problémy menších rozsahů a případné problémy s implementací
každého ze členů.

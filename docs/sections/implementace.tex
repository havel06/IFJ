\section{Implementace}
Překladač je implementovaný jako víceprůchodový.
Zvolili jsme tento typ implementace kvůli jednoduššímu členení kódu a možnosti ladit jednotlivé fáze nezávisle na sobě.
Víceprůchodový překladač se také osvědčil pro implementaci volání funkcí před jejich deklaracemi.

\subsection{Lexer}
Implementace lexeru se nachází v souborech \texttt{lexer.h} a \texttt{lexer.c}.
Primární funkcí veřejného rozhraní lexeru je funkce \texttt{getNextToken}, která zpracuje jeden token ze vstupu.
Token je struktura zestávající z typu tokenu a obsahu, reprezentovaného textovým řetězcem.
Kvůli možnosti nahlížet za následující token obsahuje lexer také buffer na jeden token, do kterého lze token vložit pomocí
funkce \texttt{unGetToken}. Při následném volání funkce \texttt{getNextToken} lexer místo zpracování vstupu vrátí token z bufferu.
Pro ladění lexeru jsme napsali pomocné funkce pro tištění tokenů, nacházející se v souborech \texttt{printToken.h} a \texttt{printToken.c}.

\subsection{Parser}
Parser je implementován v souborech \texttt{parser.h} a \texttt{parser.c}.
Používá metodu rekurzivního sestupu.
Jako vstup bere tokeny přímo z lexeru, a jako výstup sestavuje \textit{abstraktní syntaktický strom},
jehož implementaci lze nalézt v souborech \texttt{AST.h} a \texttt{AST.c}.
V souborech \texttt{printAST.h} a \texttt{printAST.c} se nachází pomocné funkce pro výpis struktury syntaktického sromu.
Pro jednodušší ošetření chyb lexeru používáme makra (např. \texttt{GET\_TOKEN}),
která při chybě lexeru automaticky vracejí z aktuální funkce a propagují chybovou hodnotu.
%TODO - expression parser

\subsection{Sémantická analýza}
Sémantický analyzátor je implementován v souborech \texttt{analyser.h} a \texttt{analyser.c}.
Na vstup bere sestavený syntaktický strom, a rekurzivním průchodem kontroluje platnost sémantických pravidel.
Syntaktický strom nijak nemodifikuje.
Nad stromem vykoná celkem dva průchody; v prvním jen zaregistruje deklarace funkcí (kvůli možnosti jejich volání před deklarací),
zatímco v druhém průchodu kontroluje vše ostatní.
Analyzátor používá tabulku symbolů pro funkce a zásobník tabulek symbolů pro proměnné.
Pro jednodušší přístup z funkcí analyzátoru jsou všechny tabulky statické globální proměnné.
Výstupem analyzátoru je zmíněná tabulka funkcí, kterou dále používá generátor cílového kódu.

\subsubsection{Implementace tabulky symbolů}
%TODO

\subsection{Generátor cílového kódu}
Generátor kódu je implementován v souborech \texttt{compiler.h} a \texttt{compiler.c}.
Na vstup bere syntaktický strom, u kterého očekává, že je sémanticky platný, a tabulku funkcí vygenerovanou analyzátorem.
Vyhodnocování složitých výrazů probíhá za pomoci zásobníku; výsledky podvýrazů vždy skončí na vrcholu zásobníku, odkud je zpracuje nadřazený výraz.
Pomocí zásobníku také předáváme parametry funkcím.
Zavolání funkce také vždy vytvoří nový rámec, díky čemuž nedochází k redeklaraci lokálních proměnných při rekurzi.
Jedním z problémů při překladu byly proměnné uvnitř cyklů, což způsobovalo redeklaraci a následně chybu interpretu.
Překladač tento problém reší deklarací proměnných před cyklem, a přeměnou všech deklarací uvnitř cyklu na jednoduchá přiřazení.
Vestavěné funkce jsou pro jednoduchost generované až v místě použití.
Díky tomu také lze jednoduše řešit volání vestavěné funkce \texttt{write}, která může přijmout libovolný počet parametrů.
